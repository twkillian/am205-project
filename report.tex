\documentclass{article}
\usepackage{amsmath, graphicx, amsfonts}
\newcommand{\R}{\mathbb{R}}

\begin{document}
\title{Doing Things With Optimization (Working Title)}
\author{Jonathan Friedman \& Taylor Killiam}
\maketitle

\section{Introduction} \label{introduction}
Serving the maximum number of people at minimum expense is an important and general problem in industry. In this paper, we examine a subproblem of the case that arises when people receive service with quality proportional to their distance from a point of service. In particular, we study the problem of finding and placing the minimum number of points of service while still achieving a given level of service per person. Store placement provides an example of this rather general formulation. A person receives "service" proportional to their distance from a store, with people becoming increasingly unlikely to shop at a store as their distance from it increases. Thus, a company must balance placing their stores near as many people as possible with the cost associated with opening and maintaining a large number of stores.

This problem is easy to solve for one point of service. In section \ref{onepoint}, we give a solution for one point using compass search. However, compass search is not viable for a large number of points because the number of required directions to search grows exponentially. Instead, we use a clustering-based algorithm. TODO: Add more detailed explanation of what we're doing.

In section \ref{experiments}, we test our algorithm by optimizing placement of points of service in Massachusetts with population data from the U.S. 2010 Census \cite{census}.

\section{One Point of Service} \label{onepoint}
Compass search is a simple method of minimizing a function $f \in \R^n$. The gist of the algorithm is as follows (see \cite{survey} for details):
\begin{itemize}
  \item Choose $\lambda_1,...,\lambda_n$ to be a positive spanning set of $\R^n$. Also choose a starting guess $p_0$, a step size $s_0$, a scaling constant $\alpha<1$, and a tolerance $t$.
  \item For $k=0,1,...$:
  \begin{itemize}
    \item If $f(p_k + s_k\lambda_i) < f(p_k)$ for some $\lambda_i$, then set $p_{k+1} = p_k + s_k\lambda_i$ and set $s_{k+1} = s_k$.
    \item Otherwise, set $p_{k+1} = p_{k}$ and set $s_{k+1} = \alpha s_k$. If $s_{k+1} < t$, return $p_{k+1}$ as the minimum.
  \end{itemize}
\end{itemize}

TODO: Add figure of compass search solution, more detailed explanation of how it doesn't work well when number of points is large.
\section{Experiments} \label{experiments}
TODO: Describe we collected, parsed, subsampled, etc. census data.

\begin{thebibliography}{9}
  \bibitem{census}
  U.S. Census Bureau. \textit{TIGER/Line� with Selected Demographic and Economic Data: Population \& Housing Unit Counts}, 2010 Census. Nov. 9, 2015. https://www.census.gov/geo/maps-data/data/tiger-data.html.
  \bibitem{survey}
  Cite survey on optimization here
\end{thebibliography}
\end{document}